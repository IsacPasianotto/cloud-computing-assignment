\section{Exercise 1}

\subsection{Requirements}
\label{subsec:requirements}

The \href{https://github.com/IsacPasianotto/cloud-computing-assignment/blob/main/exercise01/assignment.md}{assignment} requires the deployment of a scalable and secure file storage system. In particular the requirements are:

\begin{enumerate}
    \item \textit{Management of user authentication and authorization}
    \begin{enumerate}
        \item User should be able to sing up, log in and log out [\ref{subsec:nextcloud}]
        \item User should have different roles, i.e. admin and regular user [\ref{subsec:nextcloud}]
        \item Admin should be able to create, delete and modify users [\ref{subsec:nextcloud}]
        \item Regular user should have access to a personal storage space [\ref{subsec:nextcloud}]
    \end{enumerate}

    \item \textit{Management of file operations}
    \begin{itemize}
        \item User should be able to upload, download, delete and modify from their private storage space [\ref{subsec:nextcloud}]
    \end{itemize}

    \item \textit{Scalability}
    \begin{enumerate}
        \item The designed system should handle a growing number of users and files [\ref{subsec:scalability}]
        \item A theoretical discussion about the way to increase load and traffic [\ref{subsec:scalability}, \ref{subsec:costs}]
    \end{enumerate}

    \item{Security}
    \begin{enumerate}
        \item The secure file storage and transmission should be implemented [\ref{subsec:security}]
        \item Discussion about how to secure user authentication [\ref{subsec:security}, \ref{subsec:nextcloud}]
        \item Discussion about measures to prevent unauthorized access [\ref{subsec:security}]
    \end{enumerate}

    \item{Cost-effectiveness}
    \begin{itemize}
        \item Discussion about the costs implications of the designed system and how to optimize them [\ref{subsec:costs}]
    \end{itemize}

    \item{Deployment}
    \begin{enumerate}
        \item Provision of containerized environment based on docker [\ref{subsec:deployment}]
        \item Choice of a cloud provider to deploy the system in a production scenario [\ref{subsec:deployment-real-world}]
    \end{enumerate}

    \item{Testing}
    \begin{enumerate}
        \item Assessment of the system performance: load [\ref{subsubsec:test}]
        \item Assessment of the system performance: I/O  [\ref{subsubsec:test}]
    \end{enumerate}
\end{enumerate}

Following the assignment suggestion, the presented solution will be based on the \href{https://nextcloud.com/}{Nextcloud} platform.


\subsection{Technical deployment details}
\label{subsec:deployment}

The deployment of the system is based on the use of \href{https://www.docker.com/}{Docker} containers.
In particular, since more than one container is used, the \href{https://docs.docker.com/compose/}{\texttt{docker-compose}} tool is used to manage the deployment.
The complete of the multi-container setting configuration is done in the \href{https://github.com/IsacPasianotto/cloud-computing-assignment/blob/main/exercise01/docker-compose.yaml}{\texttt{docker-compose.yaml}} file.
All the user container are taken from the \href{https://hub.docker.com/}{Docker Hub} and properly configured, no custom images are used.

All the container shares the same network called \texttt{nextcloud\_network}, which can be created with:

\begin{lstlisting}[language=bash]
$ docker network create nextcloud_network
\end{lstlisting}

\subsubsection{Used images}
The \texttt{docker-compose.yml} file uses the following images\footnote{This configuration is heavily inspired by one of the \href{https://github.com/nextcloud/docker/blob/master/.examples/docker-compose/with-nginx-proxy/mariadb/fpm/docker-compose.yml}{examples provided by the Nextcloud team}.}:

\begin{enumerate}
    \item \href{https://hub.docker.com/layers/library/nextcloud/28.0.2-fpm/images/sha256-dc1b232c39cd29fe81442f0e4d1c523148afecaf0bcf1cdffb7c52441bf63af7?context=explore}{\texttt{nextcloud:28.0.2-fpm}}: 
    base image for the Nextcloud platform. In particular two images are used, the principal one for having the desired storage platform and a second one which is going to run the built-in \texttt{cron.sh}  script for background tasks\footnote{This solution was inspired by \href{https://help.nextcloud.com/t/nextcloud-docker-container-best-way-to-run-cron-job/157734/2}{this} and \href{https://help.nextcloud.com/t/clarification-regarding-cron-jobs-setup-config/134450}{this} forum pages}.
    \item \href{https://hub.docker.com/_/mariadb}{\texttt{mariadb:11.2.3}}: the databases the official Nextcloud documentation suggests to use. 
    \item \href{https://hub.docker.com/_/nginx}{\texttt{nginx:1.25.4-alpine3.18}}: the web server used to serve the Nextcloud platform, the other alternative were apache.
    \item \href{https://hub.docker.com/_/redis}{\texttt{redis:7.2.4-alpine}}: the cache server; when a client requests a file, the server will first check if the file is in the cache, this should enhance the performance.
    \item \href{https://hub.docker.com/_/caddy}{\texttt{caddy:2.7.6-alpine}}: a lightweight web server, used as a reverse proxy. It was chosen because it automatically manages the SSL certificate[\ref{subsec:security}].
\end{enumerate}

Three volumes are used to store the data: \texttt{nextcloud\_data}, \texttt{db\_data} and \texttt{caddy\_data}. 
To run the project, first create those volumes with: \texttt{docker volume <volume\_name>}, then run:

\begin{lstlisting}[language=bash]
$ docker-compose up -d
\end{lstlisting}

The \texttt{docker-compose} tool will create the containers and the network, and the system will be ready to use and accessible at \texttt{https://localhost}\footnote{A self-signed certificate is used, so the browser will show a warning.}; the first time the system is accessed, the user will be asked to create an admin account.

\subsection{Nextcloud characteristics}
\label{subsec:nextcloud}

Nextcloud offers out-of-the-box all the functionalities required at \hyperref[subsec:requirements]{points 1 and 2} of the requirements.

Using the provided web interface, the admin can create, delete and modify users.
Obviously, every created user can log in, log out to the system.
Every user (normal or admin) has access to a personal storage space, where they can upload, download, delete and modify files that are not accessible by other users.

The admin can also limited the storage space available to each user (default no limitation, but 1GB, 5GB, 15GB quotas are available).

\subsection{Scalability considerations}
\label{subsec:scalability}

\subsection{Security measures}
\label{subsec:security}

\subsection{Costs and technical considerations}
\label{subsec:costs}

\subsubsection{Test of the infrastructure}
\label{subsubsec:test}
%% load and I/O tests

\subsection{Deployment plans for hypothetical real-world scenario}
\label{subsec:deployment-real-world}

\subsection{Conclusions}
\label{subsec:conclusions}
